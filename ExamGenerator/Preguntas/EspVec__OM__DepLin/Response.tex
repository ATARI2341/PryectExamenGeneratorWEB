
<p>

    Considere los siguientes tres vectores de $\mathbb{R}^3$.

    $$
    v_1=\begin{bmatrix} 2 \\ -1 \\ 0 \end{bmatrix} \ \ \ 
    v_2=\begin{bmatrix} 0 \\ -1 \\ 2 \end{bmatrix} \ \ \ 
    v_3=\begin{bmatrix} 2 \\ 0 \\ -2 \end{bmatrix} \ \ \ 
    $$

    Haga un planteamiento claro para determinar cuál de las siguientes incisos es correcto.
    
</p>
<ol type="A" style="list-style-type: upper-alpha;">
    <li>Los tres vectores son linealmente dependientes y son una base para $\mathbb{R}^3$.</li>
<li>Los tres vectores son linealmente independientes y la única solución a $\alpha_1v_1+\alpha_2v_2+\alpha_3v_3=0$ es la trivial ($\alpha_1=0$, $\alpha_2=0$, $\alpha_3=0$).</li>
<li><b>
    Los tres vectores son linealmente dependientes, pues una solución no trivial a la ecuación $\alpha_1v_1+\alpha_2v_2+\alpha_3v_3=0$ es, por ejemplo, con $\alpha_1=1$, $\alpha_2=-1$, y $\alpha_3=-1$.
    </b></li>
<li>Los tres vectores son linealmente independientes y el vector $v=\begin{bmatrix} 2 \\ -3 \\ 4 \end{bmatrix}$ se expresa como $v=1v_1+2v_2+0v_3$.</li>
<li>Los tres vectores son linealmente independientes y son una base para $\mathbb{R}^3$.</li>
<li>No sé.</li>
</ol>
