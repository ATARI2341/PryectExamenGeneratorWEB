\documentclass[addpoints]{exam}
\usepackage[utf8]{inputenc}
\usepackage{amsmath}
\usepackage{amsfonts}
\usepackage{amssymb}
\usepackage{graphicx}
\usepackage{ctable}
\usepackage{xcolor}
\usepackage{url}
\usepackage{mathrsfs}

\def\Red#1{\textcolor{red}{#1}}
\def\Blue#1{\textcolor{blue}{#1}}

\chqword{Pregunta}
\chpgword{Página}
\chpword{Puntos}
\chbpword{Puntos extra}
\chsword{Puntos obtenidos}
\chtword{Total}

\usepackage{hyperref}


\pointpoints{punto}{puntos}
\renewcommand{\solutiontitle}{\textbf{Respuesta: }}

\CorrectChoiceEmphasis{\color{red}}

\printanswers

\begin{document}
\begin{center}
{\Large Álgebra Lineal - AD2024
\\
\vspace{5mm}
Dr. David Gómez Gutiérrez}
\vspace{0.5cm}

\fbox{\fbox{\parbox{5.5in}{\centering

\input{Instrucciones}
% \begin{itemize}
% \item \Red{Lea cuidadosamente las instrucciones.}
% \item Responda las siguientes preguntas en su cuaderno, hoja en blanco o en word, de forma clara y ordenada. 
% \item \Red{El examen es estrictamente individual}. Queda prohibido el uso de software.
% %\item Pueden reunirse en equipos de 3 personas para resolver dudas, pero cada quien debe entregar su actividad.
% \item Deberá capturar las respuestas correctas en el formulario de Google de la Actividad.
% \item Verifique sus respuestas.
% \item Deberás subir un pdf con los procedimientos en un archivo de menos de 10Mb. Recuerda que no se obtienen puntos si el procedimiento no es claro y ordenado.
% \item Se consideran respuestas correctas con un error de $\pm 0.01$.
% \end{itemize}
}}}
\end{center}

\vspace{5mm}

\makebox[\textwidth]{Student Name:\enspace\hrulefill}

\vspace{5mm}


\begin{center}
\pointtable[h][questions]
\end{center}

\pagebreak

\begin{questions}
        
\input{Questions}   
            
\end{questions}
\end{document}
        